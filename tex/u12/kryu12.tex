%\documentclass[12pt,a4paper]{article}
%\documentclass[10pt,a4paper]{report}

\documentclass[12pt,a4paper]{scrartcl}
%\documentclass[12pt,a4paper]{scrreprt}

\usepackage{amsmath,amssymb}
\usepackage[T1]{fontenc}    
\usepackage[utf8]{inputenc}
\usepackage[ngerman]{babel}
\usepackage{graphicx}

\title{Übungsblatt 12 zur Kryptographie und Kodierungstheorie}
\author{Michael Heep (1272000)}
\date{\today}

\begin{document}

\maketitle
%\tableofcontents

\section*{Aufgabe 1}
\begin{enumerate}
\item[a)]
Mit $\text{d}(C)=d \Rightarrow \text{d}(x,y) \geq d$ erhält man:
\begin{align*}
\sum_{x,y \in C} \text{d}(x,y) \geq \sum_{x,y \in C} d \\
\end{align*}
Die Menge aller Paare von Kodewörten ist $|C \times C| = |C|^2 = m^2$. Es sind jedoch $|C|=m$ viele Paare der Gestalt $(x,x)$ darunter, für die $\text{d}(x,x)=0$ gilt. Es folgt:
\begin{align*}
\sum_{x,y \in C} d = & \left( |C \times C| - |\lbrace (x,x) | x \in C \rbrace| \right) d \\
= & \left( |C|^2-|C| \right)d \\
= & \left( m^2 - m\right) d \\
= & m \left( m - 1 \right) d
\end{align*}
woraus die Behauptung folgt.
\item[b)]
$w_i$ bezeichnet die Anzahl aller Kodewörter, die an der i-ten Stelle eine 1 stehen haben. Es gibt demnach $|C|-w_i=m-w_i$ viele Kodewörter, die sich an eben dieser Stelle unterscheiden und eine 0 dort haben und für die Summe der Hamming-Abstände jeweils $+1$ bedeuten. Nun gibt es $w_i \cdot (m-w_i)$ solche Paare für die i-te Stelle. Das Summe aller Gewichte lässt sich also dann über folgende Summe berechnen:
\begin{align*}
    \sum_{i=1}^{n} w_i(m-w_i)
\end{align*}
\end{enumerate}

\section*{Aufgabe 2}

\section*{Aufgabe 3}

\end{document}